\section{Program gto\char`_real\char`_to\char`_binary\char`_with\char`_threshold}
The \texttt{gto\char`_real\char`_to\char`_binary\char`_with\char`_threshold} converts a sequence of real numbers into a binary sequence, given a threshold. The numbers below to the threshold will be 0.\\
For help type:
\begin{lstlisting}
./gto_real_to_binary_with_threshold -h
\end{lstlisting}
In the following subsections, we explain the input and output paramters.

\subsection*{Input parameters}

The \texttt{gto\char`_real\char`_to\char`_binary\char`_with\char`_threshold} program needs two streams for the computation, namely the real sequence as input. These numbers should be splitted by lines.\\
The attribution is given according to:
\begin{lstlisting}
Usage: ./gto_real_to_binary_with_threshold [options] [[--] args]
   or: ./gto_real_to_binary_with_threshold [options]

It converts a sequence of real numbers into a binary sequence given a threshold.

    -h, --help                show this help message and exit

Basic options
    -t, --threshold=<dbl>     The threshold in real format
    < input.num               Input numeric file (stdin)
    > output.bin              Output binary file (stdout)

Example: ./gto_real_to_binary_with_threshold -t <threshold> < input.num > output.bin
\end{lstlisting}
An example of such an input file is:
\begin{lstlisting}
12.25
1.2
5.44
5.51
7.97
2.34
8.123
\end{lstlisting}

\subsection*{Output}
The output of the \texttt{gto\char`_real\char`_to\char`_binary\char`_with\char`_threshold} program is a binary sequence.\\
Using the input above with the threshold of 5.5, an output example for this is the following:
\begin{lstlisting}
1
0
0
1
1
0
1
\end{lstlisting}