\section{Program gto\char`_max}
The \texttt{gto\char`_max} computes the maximum value in each row between two files.\\
For help type:
\begin{lstlisting}
./gto_max -h
\end{lstlisting}
In the following subsections, we explain the input and output paramters.

\subsection*{Input parameters}

The \texttt{gto\char`_max} program needs two streams for the computation, namely the input, which are two decimal files.\\
The attribution is given according to:
\begin{lstlisting}
Usage: ./gto_max [options] [[--] args]
   or: ./gto_max [options]

It computes the maximum value in each row between two files.

    -h, --help                Show this help message and exit

Basic options
    -f, --first_file=<str>    File to compute the max
    -s, --second_file=<str>   The second file to do the max computation
    > output.num              Output numeric file (stdout)

Example: ./gto_max -f input1.num -s input2.num > output.num
\end{lstlisting}
An example of such an input files are:\\
File 1:
\begin{lstlisting}
0.123
3.432
2.341
1.323
7.538
4.122
0.242
0.654
5.633
\end{lstlisting}
File 2:
\begin{lstlisting}
2.123
5.312
2.355
0.124
1.785
3.521
0.532
7.324
2.312
\end{lstlisting}

\subsection*{Output}
The output of the \texttt{gto\char`_max} program is the numeric file with the maximum value for each row between both input files.\\
Executing the application with the provided input, the output of this execution is:
\begin{lstlisting}
2.123000
5.312000
2.355000
1.323000
7.538000
4.122000
0.532000
7.324000
5.633000
\end{lstlisting}