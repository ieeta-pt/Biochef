\section{Program gto\char`_fastq\char`_minimum\char`_quality\char`_score}
The \texttt{gto\char`_fastq\char`_minimum\char`_quality\char`_score} discards reads with average quality-score below of the defined.\\
For help type:
\begin{lstlisting}
./gto_fastq_minimum_quality_score -h
\end{lstlisting}
In the following subsections, we explain the input and output paramters.

\subsection*{Input parameters}

The \texttt{gto\char`_fastq\char`_minimum\char`_quality\char`_score} program needs two streams for the computation, namely the input and output standard. The input stream is a FASTQ file.\\
The attribution is given according to:
\begin{lstlisting}
Usage: ./gto_fastq_minimum_quality_score [options] [[--] args]
   or: ./gto_fastq_minimum_quality_score [options]

It discards reads with average quality-score below value.

    -h, --help            show this help message and exit

Basic options
    -m, --min=<int>       The minimum average quality-score (Value 25 or 30 is commonly used)
    < input.fastq         Input FASTQ file format (stdin)
    > output.fastq        Output FASTQ file format (stdout)

Example: ./gto_fastq_minimum_quality_score -m <min> < input.fastq > output.fastq

Console output example:
<FASTQ non-filtered reads>
Total reads    : value
Filtered reads : value
\end{lstlisting}
An example of such an input file is:
\begin{lstlisting}
@SRR001666.1 071112_SLXA-EAS1_s_7:5:1:817:345 length=72
GGGTGATGGCCGCTGCCGATGGCGTCAAATCCCACCAAGTTACCCTTAACAACTTAAGGGTTTTCAAATAGA
+SRR001666.1 071112_SLXA-EAS1_s_7:5:1:817:345 length=72
IIIIIIIIIIIIIIIIIIIIIIIIIIIIII9IG9ICIIIIIIIIIIIIIIIIIIIIDIIIIIII>IIIIII/
@SRR001666.2 071112_SLXA-EAS1_s_7:5:1:801:338 length=72
GTTCAGGGATACGACGTTTGTATTTTAAGAATCTGAAGCAGAAGTCGATGATAATACGCGTCGTTTTATCAT
+SRR001666.2 071112_SLXA-EAS1_s_7:5:1:801:338 length=72
54599<>77977==6=?I6IBI::33344235521677999>>><<<@@A@BBCDGGBFFH>IIIII-I)8I
\end{lstlisting}

\subsection*{Output}
The output of the \texttt{gto\char`_fastq\char`_minimum\char`_quality\char`_score} program is a set of all the filtered FASTQ reads, followed by the execution report.\\
Using the input above with the minimum averge value as 30, an output example for this is the following:
\begin{lstlisting}
@SRR001666.1 071112_SLXA-EAS1_s_7:5:1:817:345 length=72
GGGTGATGGCCGCTGCCGATGGCGTCAAATCCCACCAAGTTACCCTTAACAACTTAAGGGTTTTCAAATAGA
+
IIIIIIIIIIIIIIIIIIIIIIIIIIIIII9IG9ICIIIIIIIIIIIIIIIIIIIIDIIIIIII>IIIIII/
Total reads    : 2
Filtered reads : 1
\end{lstlisting}
